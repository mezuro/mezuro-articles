\documentclass[final,hyperref={pdfpagelabels=false}]{beamer}
\usepackage{grffile}
\mode<presentation>{\usetheme{PosterLogProb}}
%\mode<presentation>{\usetheme{I6pd2}}
\usepackage[english]{babel}
\usepackage[latin1]{inputenc}
\usepackage{amsmath,amsthm, amssymb, latexsym}
\usepackage{epsfig}
\usepackage[orientation=portrait,size=a0,scale=1.4,debug]{beamerposter}
% change list indention level
% \setdefaultleftmargin{3em}{}{}{}{}{}
\providecommand\thispdfpagelabel[1]{}
\setbeamertemplate{bibliography entry title}{\color{black}}
\setbeamertemplate{bibliography entry location}{}
\setbeamertemplate{bibliography entry note}{}

\usepackage{array,booktabs,tabularx}
\newcolumntype{Z}{>{\centering\arraybackslash}X} % centered tabularx columns
\newcommand{\pphantom}{\textcolor{ta3aluminium}} % phantom introduces a vertical space in p formatted table columns??!!

\listfiles

\usepackage{amsthm}
%%%%%%%%%%%%%%%%%%%%%%%%%%%%%%%%%%%%%%%%%%%%%%%%%%%%%%%%%%%%%%%%%%%%%%%%%%%%%%%%%%%%%%

 \title{\huge\bfseries\hspace*{-1em} Mezuro: Understanding source code metrics}
\date{}
\author{\large Dylan Guedes
\and Paulo Meirelles
\and Rafael Manzo
\and Diego Camarinha
}

\institute{University of S�o Paulo and University of Bras�lia}

%%%%%%%%%%%%%%%%%%%%%%%%%%%%%%%%%%%%%%%%%%%%%%%%%%%%%%%%%%%%%%%%%%%%%%%%%%%%%%%%%%%%%%
\newlength{\columnheight}
\setlength{\columnheight}{105cm}

%%%%%%%%%%%%%%%%%%%%%%%%%%%%%%%%%%%%%%%%%%%%%%%%%%%%%%%%%%%%%%%%%%%%%%%%%%%%%%%%%%%%%%

\begin{document}

\begin{frame}
  \begin{columns}
    % ---------------------------------------------------------%
    % Set up a column 
    \begin{column}{.49\textwidth}
      \begin{beamercolorbox}[center,wd=\textwidth]{postercolumn}
        \begin{minipage}[T]{.95\textwidth}  % tweaks the width, makes a new \textwidth
          \parbox[t][\columnheight]{\textwidth}{ % must be some better way to set the the height, width and textwidth simultaneously
            % Since all columns are the same length, it is all nice and tidy.  You have to get the height empirically
            % ---------------------------------------------------------%
            % fill each column with content

              \begin{block}{What is Mezuro?}
    \begin{figure}
        \begin{center}
            \includegraphics[width=\textwidth]{figures/MezuroHome.png}
            \label{fig:feature1}
        \end{center}
    \end{figure}

    \begin{itemize}
        \item FOSS web-based platform that analyzes source code

        \item It allows developers to compare analyzed projects and share
            knowledge about metrics

        \item Goal to help software developers and maintainers to understand
            and feel comfortable using source code metrics during their projects

        \item Results are public

        \item Extensible architecture

        \item Currently supports Ruby, PHP, Java, C/C++, and Python

    \end{itemize}
\end{block}


%-------------------------------------------------------------------------------

            \begin{block}{Why Mezuro?}

    \begin{itemize}
        \item 100\% FOSS solution.

        \item Continuously generate reviews about a project, via customized
            scheduling.

        \item It allows a level of metrics configuration and composition that
            is not available in other tools analyzed.

        \item Related tools such as SonarQube and Code Climate:
            \begin{itemize}
                \item Are not complete FOSS.
                \item Do not allow the same metric configuration that is
                    available in Mezuro - Mezuro even allows the control of
                    the formula used in the evaluation.
                \item Do not allow processing scheduling.
            \end{itemize}

    \end{itemize}
\end{block}



%-------------------------------------------------------------------------------
            \begin{block}{Mezuro Architecture}
    \begin{itemize}
        \item Mezuro is composed of three parts: the configuration prior to
            the analysis; (ii) source code metrics computation and evaluation;
            (iii) and a graphic interface to present results.

        \item Currently, both the computation and visualization modules use
            other tools developed in the Mezuro project: \textbf{Kalibro} and
            \textbf{Prezento}.

        \item Mezuro architecture evolved to a microservice architecture, to:
            minimize the amount of code to maintain; test and grant quality of
            code; and modularize the application in several independent services.

        \item \textbf{Kalibro} is the base of the architecture, and its
            segmented in three smaller entities:

            \begin{itemize}
                \item \textbf{Kalibro Processor}, responsible for processing and
                    evaluating metrics;
                \item \textbf{Kalibro Configurations}, responsible for metrics
                    definitions and configurations;
                \item \textbf{Kalibro Client}, responsible for interoperate
                    communications between these entities.
            \end{itemize}

        \item \textbf{Prezento}, the presentation layer, mainly communicates with
            Kalibro Processor and Kalibro Configurations, via the Kalibro
            Client interface.
        \begin{figure}
            \begin{center}
                \includegraphics[scale=1.5]{figures/MezuroArchitecture.png}
                \caption{Mezuro Architecture.}
                \label{fig:architecture}
            \end{center}
        \end{figure}
    \end{itemize}
\end{block}


\vfill
        

}
\end{minipage}
\end{beamercolorbox}
\end{column}
% ---------------------------------------------------------%
% end the column

% ---------------------------------------------------------%
% Set up a column 
\begin{column}{.49\textwidth}
  \begin{beamercolorbox}[center,wd=\textwidth]{postercolumn}
    \begin{minipage}[T]{.95\textwidth} % tweaks the width, makes a new \textwidth
      \parbox[t][\columnheight]{\textwidth}{ % must be some better way to set the the height, width and textwidth simultaneously
        % Since all columns are the same length, it is all nice and tidy.  You have to get the height empirically
        % ---------------------------------------------------------%
        % fill each column with content


          \begin{block}{Project Features}
    \begin{itemize}
        \item Retrieve and evaluate source code from source code managers
            such as Git and Subversion
        \item Source code analysis scheduling
        \item Use of multiple metric configurations for multiple
            projects
        \item Visualization of historical results via charts
        \item Public results that are accessible by the community
        \item Extensible architecture to include new metric collectors
            Currently, the following collectors are available:
            \begin{itemize}
							\item Analizo (Java, C and C++)
							\item metric\_fu (Ruby)
							\item Radon (Python)
							\item CodeClimatePHPMD (PHP)
            \end{itemize}
    \end{itemize}
\end{block}

\begin{block}{Project: Metric evaluation results}
    \begin{figure}
        \begin{center}
            \includegraphics[width=\textwidth]{figures/MetricProcessing.png}
                \label{fig:feature1}
        \end{center}
    \end{figure}
\end{block}

\begin{block}{Metric Configurations Features}
    \begin{itemize}
        \item Metric configurations are responsible for allowing
            definitions and spread of metrics configurations, being
            one of Mezuro stand out factors among the others platforms
            \begin{itemize}
                \item Custom selection of metrics
                \item Statistics about most popular configurations in the
                    community
                \item Creation of ``reading groups'' to be used in textual
                    interpretation of metric results
                \item Configuration of qualitative ranges associated with the
                    value of metrics
                \item Combination of native metrics to create composed and more
                    complex analysis
            \end{itemize}
    \end{itemize}
\end{block}

\begin{block}{Configuration: Reading groups}
    \begin{figure}
        \begin{center}
            \includegraphics[width=\textwidth]{figures/ReadingGroup.png}
            \label{fig:feature1}
        \end{center}
    \end{figure}
\end{block}

\begin{block}{Next Steps}
    \begin{itemize}
        \item Front-end enhancement
        \item Improvement of processing elapsed time
        \item Auto-detection of repository language
        \item Migrate code management to Gitlab
    \end{itemize}
\end{block}

             
\vfill
      }
      % ---------------------------------------------------------%
      % end the column
        \end{minipage}
      \end{beamercolorbox}
    \end{column}
    % ---------------------------------------------------------%
    % end the column
  \end{columns}
  %\vskip1ex
  % \tiny\hfill\textcolor{ta2gray}{Created with \LaTeX \texttt{beamerposter}  \url{http://www-i6.informatik.rwth-aachen.de/~dreuw/latexbeamerposter.php}}
  
\end{frame}
\end{document}


%%%%%%%%%%%%%%%%%%%%%%%%%%%%%%%%%%%%%%%%%%%%%%%%%%%%%%%%%%%%%%%%%%%%%%%%%%%%%%%%%%%%%%%%%%%%%%%%%%%%
%%% Local Variables: 
%%% mode: latex
%%% TeX-PDF-mode: t
%%% End:

% LocalWords:  ABox ABoxes PSAT
