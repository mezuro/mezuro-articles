\section{Features}
\label{sec:features}

Mezuro features can be divided in two groups:

\begin{itemize}
    \item Project

    \begin{itemize}
        \item Obtain (download and upload) the source code via repositories (Git, Subversion, and Bazaar) or via compressed file;
        \item Code process scheduling (1 day, 2 days, weekly, biweekly, and monthly);
        \item Selection of the metric configuration should be used for each project;
        \item Visualization of measurement results per file via plot graphics with values over-time;
        \item Public results that are accessible by the community.
    \end{itemize}
    \item Configuration
    \begin{itemize}
        \item Metrics configuration cloning and creation;
        \item Statistics about most popular configurations in the community;
        \item Creation of qualitative ranges associated with values of metrics;
      	\item Creation of ``reading groups'' to be used in textual interpretation of metrics results;
        \item Combination of native metrics to create composed and more complex analysis.
    \end{itemize}
\end{itemize}

Mezuro has also a strong learning aspect, in which participants are able to
observe other projects, or clone their configurations and definitions, to learn
with them.  This open mutual interaction is interesting to project managers,
software auditors and even an entire developer team. The Mezuro main goal is to
help software developers and maintainers to understand and feel comfortable to
use source code metrics during the development cycles of their projects.

