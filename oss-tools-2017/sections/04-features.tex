\section{Features}
\label{sec:features}

Mezuro features can be divided in two groups:

\begin{itemize}
    \item Project features
    \begin{itemize}
        \item Obtain (download and upload) and evaluate source code from
            several SCM tools repositories (currently Git, Subversion, and
            Bazaar), or from compressed file;
        \item Code processing can be scheduled while preserving the obtained
            results, not forcing users to manually run the analysis from time
            to time. Currently, allows: 1 day, 2 days, weekly, and biweekly
            scheduling.
        \item Use of multiple metrics configurations for multiple projects;
        \item Visualization of measurement results per file via plot graphics
            with values over-time;
        \item Extensible architecture to include new metric collectors;
        \item Public results that are accessible by the community.
    \end{itemize}

    \item Configuration features
    \begin{itemize}
        \item Metrics creation, allowing the definition of the formula
            used to evaluate the source code;
        \item Metrics configuration cloning, giving users a quick start to rate
            their projects, and the option to use already well established
            metrics configuration;
        \item Statistics about most popular configurations in the community,
            increasing the easiness to decide what configuration to use;
        \item Creation of ``reading groups'' to be used in textual
            interpretation of metrics results;
        \item Configuration of qualitative ranges associated with values of
            metrics, allowing better control of metrics already defined;
        \item Combination of native metrics to create composed and more complex
            analysis.
    \end{itemize}
\end{itemize}

Mezuro has also a strong learning aspect, in which participants are able to
observe other projects, or clone their configurations and definitions, to learn
with them.  This open mutual interaction is interesting to project managers,
software auditors and even an entire developer team. The Mezuro main goal is to
help software developers and maintainers to understand and feel comfortable to
use source code metrics during the development cycles of their projects.

