\section{Related tools}

There are two tools related to Mezuro: SonarQube\footnote{\url{sonarqube.org}}
and Code Climate\footnote{\url{codeclimate.com}}. SonarQube offers a platform
to manage software quality by using plugins through a
library\footnote{\url{docs.sonarqube.org/display/PLUG/Plugin+Library}}.
%
At its most basic version, it classifies code problems and evaluates simple
coverage metrics in several languages. SonarQube core is licensed as LGPLv3,
however, its best plugins are paid and proprietary, such as the C/C++
analyzer\footnote{\url{sonarsource.com/why-us/products/codeanalyzers/sonarcfamilyforcpp.html}}.

Code Climate is a tool that analyzes source code hosted in a Git server, and it
has support for several programming languages and frameworks. The analysis
looks for code smells in the code and classifies them based on aspects such as
size of methods and code duplication. Based on the smells that the tool
identifies, the project under analysis will receive a grade ranging from A to
F. Keep in mind that what is tagged as a code smell sometimes might not be a
real problem, since it could be the best solution for the given scenario.

Mezuro has three main strengths regarding the tools mentioned: (i) it still
is the only complete FOSS alternative to monitor source code metrics, unlike
SonarQube, or Code Climate; (ii) it has the differential to continuously
generate reviews about the project: the user schedules the analysis and follows
metric results evolution over time; (iii) and it allows a level of metrics
configuration and composition that is not available in the other two tools,
giving control of the formula used to evaluate the metric.

These differences brings interesting scenarios for users. For instance, if a
community already knows the project requirements and has access to multiple
tools that are suitable, they can use the source code results presented by
Mezuro to lead their choice, since the data is public.
