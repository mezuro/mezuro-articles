\section{Related tools}

There are two tools related to Mezuro: SonarQube\footnote{\url{sonarqube.org}}
and Code Climate\footnote{\url{codeclimate.com}}. SonarQube offers a platform
to manage software quality by using plugins through a
library\footnote{\url{docs.sonarqube.org/display/PLUG/Plugin+Library}}.
%
At its most basic version, it classifies code problems and evaluates simple
coverage metrics in several languages. SonarQube core is licensed as LGPLv3,
however, its best plugins are paid and proprietary, such as the C/C++
analyzer\footnote{\url{sonarsource.com/why-us/products/codeanalyzers/sonarcfamilyforcpp.html}}.

Code Climate is a tool that analyzes source code hosted in a Git server, and it
has support for several programming languages and frameworks. The analysis
looks for code smells in the code and classifies them based on aspects such as
size of methods and code duplication. Based on the smells that the tool
identifies, the project under analysis will receive a grade ranging from A to
F. Keep in mind that what is tagged as a code smell sometimes might not be a
real problem, since it could be the best solution for the given scenario.

Such as SonarQube, Code Climate also is not fully FOSS and, therefore, Mezuro
still is the only complete FOSS alternative to monitor source code metrics.
%
Mezuro has the differential to continuously generate reviews about the project:
the user schedules the analysis and follows metric results evolution over time.
Results of each analysis are public, what allows greater transparency between
the developer and the community that uses the software. Thereby, the
maintainers can decide if the given solution meets the source code quality
requirements. Another example, the community already knows the project
functionally and, in a good scenario where there are more than one project with
the same functionalities, the maintainers choose the one with the best metrics.
